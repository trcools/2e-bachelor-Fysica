% Options for packages loaded elsewhere
\PassOptionsToPackage{unicode}{hyperref}
\PassOptionsToPackage{hyphens}{url}
\documentclass[
]{article}
\usepackage{xcolor}
\usepackage{amsmath,amssymb}
\setcounter{secnumdepth}{-\maxdimen} % remove section numbering
\usepackage{iftex}
\ifPDFTeX
  \usepackage[T1]{fontenc}
  \usepackage[utf8]{inputenc}
  \usepackage{textcomp} % provide euro and other symbols
\else % if luatex or xetex
  \usepackage{unicode-math} % this also loads fontspec
  \defaultfontfeatures{Scale=MatchLowercase}
  \defaultfontfeatures[\rmfamily]{Ligatures=TeX,Scale=1}
\fi
\usepackage{lmodern}
\ifPDFTeX\else
  % xetex/luatex font selection
\fi
% Use upquote if available, for straight quotes in verbatim environments
\IfFileExists{upquote.sty}{\usepackage{upquote}}{}
\IfFileExists{microtype.sty}{% use microtype if available
  \usepackage[]{microtype}
  \UseMicrotypeSet[protrusion]{basicmath} % disable protrusion for tt fonts
}{}
\makeatletter
\@ifundefined{KOMAClassName}{% if non-KOMA class
  \IfFileExists{parskip.sty}{%
    \usepackage{parskip}
  }{% else
    \setlength{\parindent}{0pt}
    \setlength{\parskip}{6pt plus 2pt minus 1pt}}
}{% if KOMA class
  \KOMAoptions{parskip=half}}
\makeatother
\setlength{\emergencystretch}{3em} % prevent overfull lines
\providecommand{\tightlist}{%
  \setlength{\itemsep}{0pt}\setlength{\parskip}{0pt}}
\usepackage{bookmark}
\IfFileExists{xurl.sty}{\usepackage{xurl}}{} % add URL line breaks if available
\urlstyle{same}
\hypersetup{
  hidelinks,
  pdfcreator={LaTeX via pandoc}}

\author{}
\date{}

\begin{document}

\section{HC19--20 --- Oplosbaarheid (Kₛ) \& gemeenschappelijk-ion-effect
(examengericht)}\label{hc1920-oplosbaarheid-kux209b-gemeenschappelijk-ion-effect-examengericht}

\subsection{\texorpdfstring{1) Wat de prof \textbf{echt} wil dat je
snapt}{1) Wat de prof echt wil dat je snapt}}\label{wat-de-prof-echt-wil-dat-je-snapt}

\begin{itemize}
\tightlist
\item
  \textbf{Oplosbaarheid is een evenwichtsprobleem.} Een slecht oplosbaar
  zout staat in evenwicht tussen vaste stof en ionen in oplossing.
\item
  \textbf{Het oplosbaarheidsproduct \(K_s\) is een evenwichtsconstante}
  (dus thermodynamisch gelinkt aan \(\Delta_r G^\circ\)): \[
  K_s = \exp\!\left(-\frac{\Delta_r G^\circ}{RT}\right)
  \]
\item
  \textbf{\(K_s\) ≠ ``oplosbaarheid''.} \(K_s\) zegt iets over het
  evenwicht, maar de \emph{molaire oplosbaarheid} \(S\) (mol/L) hangt
  ook af van:

  \begin{itemize}
  \tightlist
  \item
    \textbf{stoichiometrie} van het zout
  \item
    \textbf{samenstelling} van de oplossing (gemeenschappelijke ionen,
    complexvorming, pH, \ldots)
  \item
    \textbf{temperatuur}
  \end{itemize}
\end{itemize}

\textbf{Cruciale examenvuistregel:}\\
\textgreater{} Je mag \(K_s\) van verschillende zouten \textbf{niet}
zomaar onderling vergelijken om te zeggen welk zout het ``meest
oplosbaar'' is, tenzij je eerst de \textbf{molaire oplosbaarheid \(S\)}
uitrekent.

\begin{center}\rule{0.5\linewidth}{0.5pt}\end{center}

\subsection{2) Basisopstelling: hoe je altijd
start}\label{basisopstelling-hoe-je-altijd-start}

Neem een algemeen zout: \[
\mathrm{M_aX_b(s) \rightleftharpoons a\,M^{n+}(aq) + b\,X^{m-}(aq)}
\]

Dan (in verdunde oplossingen, met activiteiten \(\approx\)
concentraties): \[
K_s = [\mathrm{M^{n+}}]^a[\mathrm{X^{m-}}]^b
\]

\subsubsection{\texorpdfstring{Molaire oplosbaarheid \(S\) in zuiver
water}{Molaire oplosbaarheid S in zuiver water}}\label{molaire-oplosbaarheid-s-in-zuiver-water}

Laat \(S = x\) mol/L zout oplossen. Dan: - \([\mathrm{M^{n+}}] = a x\) -
\([\mathrm{X^{m-}}] = b x\)

Dus: \[
K_s = (a x)^a (b x)^b
\quad\Rightarrow\quad
x = S
\]

\textbf{Mini-cheatsheet (komt rechtstreeks uit de slides/oefeningen):} -
\$ \mathrm{MX(s)\rightleftharpoons M^+ + X^-}\$\\
\(K_s = x^2 \Rightarrow S=\sqrt{K_s}\) - \$
\mathrm{M_2X(s)\rightleftharpoons 2M^+ + X^{2-}}\$\\
\(K_s=(2x)^2(x)=4x^3 \Rightarrow S=\left(\frac{K_s}{4}\right)^{1/3}\) -
\$ \mathrm{M_2X_3(s)\rightleftharpoons 2M^{3+} + 3X^{2-}}\$\\
\(K_s=(2x)^2(3x)^3=108x^5 \Rightarrow S=\left(\frac{K_s}{108}\right)^{1/5}\)

\textbf{Waarom dit belangrijk is:} in de slides staat expliciet een
oefening waar zouten met verschillende stoichiometrie een \(K_s\)
krijgen en je \emph{moet} rangschikken op oplosbaarheid. Dat kan enkel
via \(S\).

\begin{center}\rule{0.5\linewidth}{0.5pt}\end{center}

\subsection{3) Gemeenschappelijk-ion-effect (dé
klassieker)}\label{gemeenschappelijk-ion-effect-duxe9-klassieker}

Als er al een ion aanwezig is dat in het evenwicht voorkomt, dan
verschuift het evenwicht \textbf{naar links} ⇒ \textbf{oplosbaarheid
daalt}.

\subsubsection{\texorpdfstring{Voorbeeldtype uit de slides:
\(\mathrm{Ag_2CrO_4}\)}{Voorbeeldtype uit de slides: \textbackslash mathrm\{Ag\_2CrO\_4\}}}\label{voorbeeldtype-uit-de-slides-mathrmag_2cro_4}

\[
\mathrm{Ag_2CrO_4(s) \rightleftharpoons 2Ag^+ + CrO_4^{2-}}
\] \[
K_s = [\mathrm{Ag^+}]^2[\mathrm{CrO_4^{2-}}]
\]

Stel: je hebt al \(0{,}10\ \mathrm{mol/L}\) \(\mathrm{Ag^+}\) in
oplossing, en extra oplosbaarheid is \(x\): -
\([\mathrm{Ag^+}] = 0{,}10 + 2x\) - \([\mathrm{CrO_4^{2-}}] = x\)

Dan: \[
K_s = (0{,}10+2x)^2(x)
\]

\textbf{De exametruc die de prof toont:} Als \(0{,}10 \gg 2x\), dan: \[
(0{,}10+2x)\approx 0{,}10
\quad\Rightarrow\quad
K_s \approx (0{,}10)^2 x
\Rightarrow x\approx \frac{K_s}{(0{,}10)^2}
\]

\textbf{Altijd doen op examen:} check achteraf of \(2x \ll 0{,}10\) echt
klopt.\\
Als dat niet klopt → geen benadering, dan los je exact op (vaak een
kubiek/kwadratisch, maar meestal is de benadering juist gekozen).

\begin{center}\rule{0.5\linewidth}{0.5pt}\end{center}

\subsection{4) Stappenplan dat bijna elke oefening
oplost}\label{stappenplan-dat-bijna-elke-oefening-oplost}

\begin{enumerate}
\def\labelenumi{\arabic{enumi}.}
\tightlist
\item
  \textbf{Schrijf de oplosreactie} correct (met stoichiometrie).
\item
  \textbf{Schrijf \(K_s\)} als product van ionconcentraties met machten.
\item
  \textbf{ICE/massabalans}: zet beginconcentraties + verandering door
  oplossen (\(x\)).
\item
  \textbf{Substitueer} alles in \(K_s\).
\item
  \textbf{Kies (indien mogelijk) een benadering} (bv. gemeenschappelijk
  ion domineert).
\item
  \textbf{Los op voor \(x\) (= \(S\))}.
\item
  \textbf{Consistentiecheck} van je benadering.
\end{enumerate}

\begin{center}\rule{0.5\linewidth}{0.5pt}\end{center}

\subsection{5) Typische valkuilen (waar punten
verdampen)}\label{typische-valkuilen-waar-punten-verdampen}

\begin{itemize}
\tightlist
\item
  \(K_s\)-waarden vergelijken zonder naar stoichiometrie te kijken.
\item
  Vergeten dat bij \(\mathrm{M_2X}\): \([\mathrm{M}]=2S\) en niet \(S\).
\item
  Benadering maken (``\(0{,}10+2x \approx 0{,}10\)'') en \textbf{niet}
  checken.
\item
  Denken dat ``groter \(K_s\) altijd groter \(S\)'' is --- \emph{soms},
  maar niet universeel.
\end{itemize}

\begin{center}\rule{0.5\linewidth}{0.5pt}\end{center}

\subsection{6) Wat je ``paraat'' wil hebben voor het
examen}\label{wat-je-paraat-wil-hebben-voor-het-examen}

\begin{itemize}
\tightlist
\item
  Definitie: \(K_s\) als evenwichtsconstante van oplossen.
\item
  Omzetting \(K_s \leftrightarrow S\) via stoichiometrie (voor 1:1, 2:1,
  2:3 moet je dit snel kunnen).
\item
  Gemeenschappelijk-ion-effect kunnen opstellen én benaderen.
\item
  Het mantra: \textbf{``\(K_s\) is niet oplosbaarheid; \(S\) is
  oplosbaarheid.''}
\end{itemize}

\section{HC21 -- Elektrochemie (examengerichte
samenvatting)}\label{hc21-elektrochemie-examengerichte-samenvatting}

\subsection{\texorpdfstring{1) De kernidee (wat je \emph{altijd} moet
kunnen)}{1) De kernidee (wat je altijd moet kunnen)}}\label{de-kernidee-wat-je-altijd-moet-kunnen}

Een \textbf{galvanische (volta-)cel} zet een \textbf{spontane
redoxreactie} om in \textbf{elektrische arbeid} via elektronen die door
een uitwendig circuit lopen.

Spontaniteit-criteria (cruciaal): - \textbf{Spontaan:}
\(E_\text{cel} > 0 \;\Leftrightarrow\; \Delta G < 0\) -
\textbf{Evenwicht / cel ``plat'':}
\(E_\text{cel}=0 \;\Leftrightarrow\; \Delta G=0\) en dan geldt \(Q=K\)

\textbf{Anode/Kathode (altijd examenvalkuil):} - \textbf{Anode =
oxidatie} - \textbf{Kathode = reductie} - In een \textbf{galvanische}
cel: anode is typisch \textbf{negatief}, kathode \textbf{positief}.

\subsection{2) Opbouw van een galvanische cel (Daniell/Zn--Cu is het
archetype)}\label{opbouw-van-een-galvanische-cel-daniellzncu-is-het-archetype}

\begin{itemize}
\tightlist
\item
  2 \textbf{halfcellen} (elk een elektrode + elektrolyt).
\item
  \textbf{Zoutbrug} (bv. KNO₃ of NH₄NO₃ in gel) zorgt voor ionenmigratie
  zodat elke halfcel elektrisch neutraal blijft.
\item
  \textbf{Uitwendig circuit}: elektronen lopen van anode → kathode.
\end{itemize}

\textbf{Actieve vs inerte elektroden} - Actief: de elektrode doet mee in
de redox (bv. Zn(s), Cu(s)). - Inert: geleidt enkel e⁻, doet niet mee
(bv. Pt(s), grafiet).

\subsection{3) Elektrodepotentialen en celspanning
(rekenrecept)}\label{elektrodepotentialen-en-celspanning-rekenrecept}

\textbf{Standaardreductiepotentialen \(E^\circ\)} zijn gedefinieerd
t.o.v. de \textbf{standaard-waterstofelektrode (SHE)} met
\(E^\circ = 0.000\ \text{V}\).

Belangrijkste formule: {[} E\^{}\circ\emph{\text{cel} =
E\^{}\circ}\{\text{kathode (reductie)}\} -
E\^{}\circ\_\{\text{anode (reductie)}\} {]}

\textbf{Examenschema om \(E^\circ_\text{cel}\) te vinden} 1. Schrijf
beide halfreacties als \textbf{reducties} (zoals in de tabel met
\(E^\circ\)). 2. De halfreactie met \textbf{grootste \(E^\circ\)} wordt
de \textbf{kathode} (gaat effectief als reductie). 3. De andere wordt
\textbf{anode} (gaat effectief als oxidatie; teken keert om in je
reactie, maar je gebruikt in de formule nog steeds het
\emph{reductie}-\(E^\circ\)). 4. Bereken \(E^\circ_\text{cel}\) met de
formule hierboven. 5. Balanceer elektronen voor de totale reactie, maar
let op: - \textbf{Je vermenigvuldigt \(E^\circ\) nooit met
stoichiometrische factoren.}

\subsection{4) Link met thermodynamica: arbeid, vrije energie en
evenwicht}\label{link-met-thermodynamica-arbeid-vrije-energie-en-evenwicht}

Elektrische arbeid en vrije energie hangen samen via: {[} \Delta G = -nF
E\_\text{cel} {]} waar \(n\) = aantal uitgewisselde elektronen en \(F\)
= Faraday-constante.

Standaardrelaties (super-examenwaardig): {[} \Delta G\^{}\circ = -nF
E\^{}\circ\emph{\text{cel} {]} {[} \Delta G\^{}\circ = -RT\ln K {]} {[}
E\^{}\circ}\text{cel}=\frac{RT}{nF}\ln K {]}

Interpretatie in één oogopslag: -
\(\Delta G^\circ<0 \Rightarrow K>1 \Rightarrow E^\circ_\text{cel}>0\)
(spontaan in standaardcondities) -
\(\Delta G^\circ=0 \Rightarrow K=1 \Rightarrow E^\circ_\text{cel}=0\)
(evenwicht) -
\(\Delta G^\circ>0 \Rightarrow K<1 \Rightarrow E^\circ_\text{cel}<0\)
(niet spontaan)

\subsection{5) Nernstvergelijking: niet-standaard
condities}\label{nernstvergelijking-niet-standaard-condities}

Als concentraties/drukken afwijken van standaardcondities, gebruik je
\textbf{Nernst}: {[} E = E\^{}\circ - \frac{RT}{nF}\ln Q {]} Bij
\(25^\circ\text{C}\) vaak als: {[} E = E\^{}\circ -
\frac{0.05916}{n}\log\_\{10\}Q {]}

\textbf{Wat is \(Q\)?} Het reactiequotiënt: producten/reactanten met
macht volgens stoichiometrie. - Zuivere vaste stoffen en vloeistoffen
komen \textbf{niet} in \(Q\).

\subsection{6) Concentratiecellen (klassiek
examenvraagstuk)}\label{concentratiecellen-klassiek-examenvraagstuk}

Een \textbf{concentratiecel} heeft dezelfde halfreactie links en rechts,
maar met \textbf{verschillende concentraties}. Voorbeeldnotatie: {[}
\text{Ag(s)};\textbar;\text{Ag}\textsuperscript{+(1.0,\text{M});\textbar\textbar;\text{Ag}}+(0.1,\text{M});\textbar;\text{Ag(s)}
{]}

Belangrijk: - De celspanning is typisch \textbf{klein}. - Je berekent
elke halfcelpotentiaal met \textbf{Nernst}, en dan: {[}
E\_\text{cel}=E\_\text{rechts}-E\_\text{links}\quad(\text{of consistent met kathode–anode})
{]} - De kant met ``hogere reductiepotentiaal'' fungeert als
\textbf{kathode}.

\subsection{7) Commerciële voltacellen (wat je moet herkennen +
kernreacties)}\label{commerciuxeble-voltacellen-wat-je-moet-herkennen-kernreacties}

Je hoeft dit meestal niet hyper-diep af te leiden, maar je moet het
\textbf{type, de idee en vaak de halfreacties kunnen plaatsen}.

\subsubsection{Loodaccumulator (auto, \textasciitilde12 V
totaal)}\label{loodaccumulator-auto-12-v-totaal}

\begin{itemize}
\tightlist
\item
  Ongeveer \textbf{2 V per cel}, typisch 6 cellen in serie.
\item
  Bij ontlading wordt \textbf{H₂SO₄ verbruikt} (dichtheid daalt).
\item
  Oplaadbaar: reacties omkeerbaar via externe stroombron.
\item
  Bij laden kan water-elektrolyse (H₂/O₂) optreden → veiligheidsaspect.
\end{itemize}

\subsubsection{Droge cellen (1.25--1.50
V)}\label{droge-cellen-1.251.50-v}

\begin{itemize}
\tightlist
\item
  \textbf{Leclanché-element} (klassieke ``zink-kool''-achtige cel).
\item
  \textbf{Alkalische batterij}: langere levensduur (zinkanode corrodeert
  trager in basisch milieu).
\item
  \textbf{Zilvercel}: gebruikt in kleine toestellen (uurwerken,
  pacemakers, hoorapparaten, \ldots).
\end{itemize}

\subsubsection{Nikkel--cadmium (Ni--Cd, \textasciitilde1.4
V)}\label{nikkelcadmium-nicd-1.4-v}

\begin{itemize}
\tightlist
\item
  Oplaadbaar (producten blijven aan elektroden ``kleven'' volgens de
  cursuscontext).
\item
  Toepassingen: boormachines, scheerapparaten, \ldots{}
\end{itemize}

\subsubsection{\texorpdfstring{Brandstofcel (H₂/O₂,
\(E_\text{cel}\approx 1.2\)
V)}{Brandstofcel (H₂/O₂, E\_\textbackslash text\{cel\}\textbackslash approx 1.2 V)}}\label{brandstofcel-hux2082oux2082-e_textcelapprox-1.2-v}

\begin{itemize}
\tightlist
\item
  Reagentia worden continu aangevoerd.
\item
  Nettoreactie:
  \(2\text{H}_2 + \text{O}_2 \rightarrow 2\text{H}_2\text{O}\)
\item
  Efficiëntie-idee: groot deel van theoretische \(\Delta G\) →
  elektrische energie; nadelen: opslag en dure elektroden.
\end{itemize}

\subsubsection{Li-ion (conceptueel
herkennen)}\label{li-ion-conceptueel-herkennen}

\begin{itemize}
\tightlist
\item
  \textbf{Anode: grafiet}
\item
  \textbf{Kathode: LiCoO₂}
\item
  \textbf{Li⁺-transport} tussen elektroden (intercalatie/de-intercalatie
  als idee).
\end{itemize}

\subsection{8) Examenvallen \&
mini-checklist}\label{examenvallen-mini-checklist}

\begin{itemize}
\tightlist
\item
  \textbf{Anode ≠ altijd positief}: in galvanische cel is anode typisch
  \textbf{negatief} (oxidatie), kathode positief (reductie).
\item
  \textbf{\(E^\circ\) nooit schalen met coëfficiënten}.
\item
  \textbf{Tabelwaarden zijn reductiepotentialen}: als je een oxidatie
  gebruikt, keer je de reactie om maar je werkt nog steeds met
  reductie-\(E^\circ\) in
  \(E^\circ_\text{cel}=E^\circ_\text{kath}-E^\circ_\text{an}\).
\item
  \textbf{Standaardcondities}: opgeloste species 1 M, gassen 1 bar,
  zuivere vaste stoffen/vloeistoffen activiteit 1.
\item
  Bij ``cel plat'': \textbf{\(E_\text{cel}=0\) én \(\Delta G=0\) én
  \$Q=K}.
\end{itemize}

\subsection{9) Snelle ``rekenflow'' (wat je op je kladpapier
wil)}\label{snelle-rekenflow-wat-je-op-je-kladpapier-wil}

\begin{enumerate}
\def\labelenumi{\arabic{enumi}.}
\tightlist
\item
  Identificeer halfreacties + \(E^\circ\) uit tabel.
\item
  Kies kathode = hoogste \(E^\circ\).
\item
  \(E^\circ_\text{cel}=E^\circ_\text{kath}-E^\circ_\text{an}\).
\item
  Balanceer reactie → bepaal \(n\).
\item
  Indien niet-standaard: \(E=E^\circ-\frac{RT}{nF}\ln Q\).
\item
  \(\Delta G=-nFE\) en eventueel \(K\) via
  \(E^\circ_\text{cel}=\frac{RT}{nF}\ln K\).
\end{enumerate}

\end{document}
